%!TEX TS-program = xelatex
%!TEX encoding = UTF-8 Unicode
% Awesome CV LaTeX Template
%
% This template has been downloaded from:
% https://github.com/posquit0/Awesome-CV
%
% Author:
% Claud D. Park <posquit0.bj@gmail.com>
% http://www.posquit0.com
%
% Template license:
% CC BY-SA 4.0 (https://creativecommons.org/licenses/by-sa/4.0/)
%

%%%%%%%%%%%%%%%%%%%%%%%%%%%%%%%%%%%%%%
%     Configuration
%%%%%%%%%%%%%%%%%%%%%%%%%%%%%%%%%%%%%%
%%% Themes: Awesome-CV
\documentclass[]{awesome-cv}
\usepackage{textcomp}
\usepackage[none]{hyphenat}

%%% Override a directory location for fonts(default: 'fonts/')
\fontdir[fonts/]

%%% Configure a directory location for sections
\newcommand*{\sectiondir}{resume/}

%%% Override color
% Awesome Colors: awesome-emerald, awesome-skyblue, awesome-red, awesome-pink, awesome-orange
%                 awesome-nephritis, awesome-concrete, awesome-darknight
%% Color for highlight
% Define your custom color if you don't like awesome colors
\colorlet{awesome}{awesome-red}
%\definecolor{awesome}{HTML}{CA63A8}
%% Colors for text
%\definecolor{darktext}{HTML}{414141}
%\definecolor{text}{HTML}{414141}
%\definecolor{graytext}{HTML}{414141}
%\definecolor{lighttext}{HTML}{414141}

%%% Override a separator for social information in header(default: ' | ')
%\headersocialsep[\quad\textbar\quad]
\begin{document}
%%%%%%%%%%%%%%%%%%%%%%%%%%%%%%%%%%%%%%
%     Profile
%%%%%%%%%%%%%%%%%%%%%%%%%%%%%%%%%%%%%%
\begin{center}
    \headerfirstnamestyle{Nishant} \headerlastnamestyle{Singh} \\
    \vspace{2mm}
    {\faEnvelope\ talktonishantsingh.ns@gmail.com} | {\faMobile\ +91 6352403156} | {\faMapMarker\ Noida, UP} | {\faLinkedinSquare\ \href{https://www.linkedin.com/in/tesla59/}{tesla59}} | {\faGithubSquare\ \href{https://www.github.com/tesla59/}{tesla59}} \
\end{center}
\vspace{-3mm}
%%%%%%%%%%%%%%%%%%%%%%%%%%%%%%%%%%%%%%
%     Education
%%%%%%%%%%%%%%%%%%%%%%%%%%%%%%%%%%%%%%
\cvsection{Education}
\begin{cventries}
    \cventry
    {\textbf{B.Tech in Information Technology | GPA: 8.76 }}
    {Indian Institute of Information Technology Bhopal}
    {Bhopal, MP}
    {Dec, 2021 – Present}
    % {GPA: 8.65}
    {\hspace{-22mm}\textbf{Relevant Coursework}: Data Structures and Algorithm, Operating System, Computer Network, Database Management System}
\end{cventries}
\vspace{-5mm}

%%%%%%%%%%%%%%%%%%%%%%%%%%%%%%%%%%%%%%
%     Experience
%%%%%%%%%%%%%%%%%%%%%%%%%%%%%%%%%%%%%%
\cvsection{Work Experience}
\vspace{-3mm}
\begin{cventries}
    % -2 GSoC
    \cventry
    {Mentor}
    {Google Summer of Code}
    {Remote}
    {Mar, 2025 – Present}
    {\begin{cvitems}
        \item Mentored contributors to design \& implement a scalable observability layer for KubeArmor, integrating 6+ Prometheus metrics to expose real-time insights into policy enforcement, alerting, \& overall security activity across Kubernetes \& non-Kubernetes workloads
        \item Introduced and exposed 4+ key production-grade metrics endpoints in Go, adhering to Prometheus exposition standards and enabling seamless integration with observability tools like Grafana, Alertmanager, and custom dashboards, leading to improved observability
        \item Co-ordinated closely with core maintainers and mentors over 175+ engineering hours to align implementation with CNCF open-source guidelines ensuring consistent code quality, improving traceability, policy enforcement monitoring, and ecosystem maintainability
        \vspace{-2mm}
        \end{cvitems}}
    % -1 Samsung
    \vspace{-2mm}
    \cventry
    {Software Developer Intern}
    {Samsung Research Institute - Delhi}
    {Noida, UP}
    {Jan, 2025 – May, 2025}
    {\begin{cvitems}
        % \item Enforced API request validations by defining field constraints (e.g., min/max values) in Swagger specs and implementing runtime checks using the validator package in Go across 10–12 backend APIs, contributing to robust API validation and improved reliability
        \item Implemented runtime API validation using the validator package in Go, adding field-level constraints like minimum/maximum values to Swagger specifications for 11 Iris-based backend APIs, reducing 80\% of validation errors and improving overall API reliability
        \item Revamped UMD movie matching by manually mapping unmatched metadata entries between Samsung’s UMD and IMDb, and developing a custom CLI tool that reduced the task time from 4–5 hours to just 30 minutes daily
        \end{cvitems}}
    \vspace{-2mm}
    % 0. LFX
    \cventry
    {LFX Mentee | Open Source Contributor}
    {Cloud Native Computing Foundation (CNCF)}
    {Salt Lake, USA}
    {Sept, 2024 – Nov, 2024}
    {\begin{cvitems}
        \item Transformed the security and management functionality of KubeArmor for virtual machines (VMs) by expanding its feature set by 15\% and enhancing the capabilities for unorchestrated containers and non-kubernetes environments
        \item Developed and optimized 3 real-time configuration interfaces (via karmor.yaml, gRPC, and CLI), enabling real-time updates to security settings without the need for service restarts. Increased configuration efficiency and flexibility, reducing manual intervention
        \item Streamlined the installation and policy management process by refining the KubeArmor CLI for VMs, reducing installation time by 25\%, while extending CLI capabilities to enable faster policy management, resulting in 20\% reduction in operational overhead
        \end{cvitems}}

    \vspace{-2mm}
    % 1. Gocomet
    \cventry
    {DevOps Engineer Intern}
    {GoComet Pvt Ltd}
    {Bangalore}
    {Oct, 2023 – Oct, 2024}
    {\begin{cvitems}
            \item {Established a robust monitoring infrastructure by configuring alerts for PostgreSQL 13, specifically targeting the long idling transactions, locks and max concurrent connections, providing immediate visibility into potential issues and enabling swift resolution}
            \item {Spearheaded the daily deployment process of production code through a enhanced rolling deployment strategy on Kubernetes, orchestrating seamless updates across 5+ systems and leveraging automated tools to minimize downtime \& enhance system reliability}
            \item {Enhanced operational insights by setting up Fluentd logging agent to push detailed, enriched application logs to New Relic, enabling deeper understanding of application behavior, streamlined troubleshooting the intricacies \& reducing average debugging time by 30\%}
            % \item Enhanced operational insights by setting up the Fluentd logging agent to push enriched application logs to New Relic, reducing average debugging time by 30\% and enabling deeper analysis and understanding of application behavior
        \end{cvitems}}
    % 2. Mobilics
    % \vspace{-3mm}
    % \cventry
    % {DevOps Engineer Intern}
    % {Mobilicis India Pvt Ltd}
    % {Remote}
    % {Jun, 2023 – Oct, 2023}
    % {\begin{cvitems}
    %         \item {Executed a successful comprehensive migration involving the seamless transition of 300+ repositories from Bit Bucket to GitHub}
    %         \item {Implemented comprehensive testing methodologies and conducted over 50 robust tests to guarantee the establishment of secure and enabled hierarchical access controls for the repositories and the CI/CD pipeline, thereby enhancing overall development efficiency}
    %         \item {Pioneered successful implementation of 8+ robust Jenkins-based CI/CD pipeline, automating build, test, and deployment processes}
    %     \end{cvitems}}

    % 3. Renal Project (Goodbye Soldier! You're the reason I am here today. You will always be remembered for being the stepping stone!)
    % \vspace{-3mm}
    % \cventry
    % {DevOps Engineer Intern}
    % {The Renal Project}
    % {Remote}
    % {Jul, 2022 – Apr, 2023}
    % {\begin{cvitems}
    %         \item {Led the design, implementation, and maintenance of automated CI/CD pipelines for code deployment using GitHub Actions, resulting in 30\% improvement in development and testing efficiency by strategically eliminating 70\% of manual deployment processes}
    %         \item {Administered successful migration of 4+ servers from Oracle Cloud Infra to AWS Cloud, employed Terraform to implement declarative methodologies and facilitated the adoption of AWS Elastic Kubernetes Service (EKS), replacing traditional Docker deployment}
    %     \end{cvitems}}
\end{cventries}
\vspace{-6mm}

%%%%%%%%%%%%%%%%%%%%%%%%%%%%%%%%%%%%%%
%     Projects
%%%%%%%%%%%%%%%%%%%%%%%%%%%%%%%%%%%%%%
\cvsection{Projects}
\vspace{-3.5mm}
\begin{cventries}
    % -1. Blaze
    \cventry
    {Go, NextJS, PostgreSQL, WebSocket, WebRTC}
    {Blaze}
    {\href{https://github.blazetv.live}{blazetv.live}}
    {}
    {\begin{cvitems}
            \item Coded a scalable Omeagle-like real-time video chat platform using Gorilla WebSockets in Go, achieving 10,000+ concurrent WebSocket connections on a single core, structured using SOLID principles and a clean Repository-Service-Controller architecture
            \item Leveraged PostgreSQL to persist 100,000+ user sessions, metadata, and abuse reports, while building a Next.js frontend integrated with WebRTC, supporting 5,000+ simultaneous peer-to-peer video/audio streams, coordinated through a custom Go signaling server
            \item Designed and implemented robust CI/CD pipelines using GitHub Actions, automating deployment workflows reducing release times by 70\%, and containerized the application with Docker to enable consistent and scalable deployments across multiple environments
        \end{cvitems}}

    \vspace{-4mm}
    % 0. GoLoadBalancer
    \cventry
    {Go, Docker, Load Balancer, Concurrency}
    {GoLoadBalancer}
    {\href{https://github.com/tesla59/GoLoadBalancer}{tesla59/GoLoadBalancer}}
    {}
    {\begin{cvitems}
            \item Developed an automatic load balancer in Go v1.21 with round-robin functionality, inspired by Kubernetes v1.24 operator architecture
            \item Leveraged Goroutine and Docker to spawn 3+ load balancer to balance the incoming load and 5+ workers to handle incoming request
            \item Incorporated SQLite 3 Database to store and provide real-time statistics such as of all workers such as average failure rate and delay
        \end{cvitems}}

    \vspace{-4mm}
    % 1. WhisperWeb
    \cventry
    {Go, Fiber, REST APIs, SQLite, Docker}
    {WhisperWeb}
    {\href{https://github.com/tesla59/whisperweb-backend}{tesla59/whisperweb-backend}}
    {}
    {\begin{cvitems}
            \item Coded a HTTP Server using Fiber web framework in Go for back-end using SQLite 3 as the DBMS to store and retrieve data
            \item Designed the front-end part using Next JS 13 capitalising Server-Side Rendering (SSR) pattern to enhance performance and experience
            \item Containerized the website using Docker ensuring portability and reliability across 5+ different environments and easy deployment
        \end{cvitems}}

    % \vspace{-5mm}
    % 2. Shell.js
    % \cventry
    % {Node JS, Linux, Bash, REPL}
    % {Shell JS}
    % {\href{https://github.com/tesla59/shell.js}{tesla59/shell.js}}
    % {}
    % {\begin{cvitems}
    %         \item Innovated and developed a custom Bash-like Shell using Node.js version 16, using built-in asynchronous libraries in JavaScript
    %         \item Redesigned asynchronous libraries to optimise performance, enabling concurrent execution of commands like CD, LS, PWD etc
    %     \end{cvitems}}

    % \vspace{-5mm}
    % 3. Hydra Kernel
    % \cventry
    % {C, Linux, Linux Kernel}
    % {Hydra Kernel}
    % {\href{https://github.com/tesla59/hydra_kernel}{tesla59/hydra\textunderscore{}kernel}}
    % {}
    % {\begin{cvitems}
    %         \item Acted as an Open-Source Developer and modified the Android Kernel based on v4.14 for Redmi Note 7 Pro for enhanced efficiency
    %         \item Increased performance by 20\% in benchmark and reduced battery consumption by 10\% in daily usage in comparison to stock kernel
    %     \end{cvitems}}


    \vspace{-7mm}
\end{cventries}

%%%%%%%%%%%%%%%%%%%%%%%%%%%%%%%%%%%%%%
%     Skills
%%%%%%%%%%%%%%%%%%%%%%%%%%%%%%%%%%%%%%
\cvsection{Skills}
\vspace{-5mm}
\begin{cventries}
    \cventry
    {}
    {
        \def\arraystretch{1.15}
        {
            \begin{tabular}{ l l }
                Languages:  & {\skill{ Go, Rust, TypeScript, JavaScript, Dart, Python, C, C++}}        \\
                Tools:      & {\skill{ Docker, Kubernetes, Terraform, Ansible, Git, ArgoCD, SQL, Jenkins}}           \\
                Frameworks: & {\skill{ Iris, Gorilla, Flutter, NextJS, GoFiber, Gorm}}                  \\
                Platforms:  & {\skill{ Linux, AWS Cloud, Azure, Google Cloud Platform, Oracle Cloud Infrastructure}} \\
            \end{tabular}}}
    {}
    {}
    {}
\end{cventries}
\vspace{-12mm}

%%%%%%%%%%%%%%%%%%%%%%%%%%%%%%%%%%%%%%
%     Certificates
%%%%%%%%%%%%%%%%%%%%%%%%%%%%%%%%%%%%%%
\cvsection{Certificates}
\begin{cvhonors}
    \cvhonor
    {Google Cloud Certified Professional Cloud Architect}
    {}
    % {Mastered certification for the Google Cloud Professional Cloud Architect exam, showcasing expertise in architecting secure and complex solutions with high scalability and availability on the Google Cloud Platform}
    {Google Cloud}
    {Jan, 2024}
    % \cvhonor
    % {Google Cloud Certified Associate Cloud Engineer}
    % {Achieved certification as a Google Cloud Associate Cloud Engineer, showcasing proficiency in designing, implementing and managing cloud solutions on the Google Cloud Platform and mastery in skills within Google Cloud}
    % {Google Cloud}
    % {Nov, 2023}
\end{cvhonors}
\
\end{document}
