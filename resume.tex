%!TEX TS-program = xelatex
%!TEX encoding = UTF-8 Unicode
% Awesome CV LaTeX Template
%
% This template has been downloaded from:
% https://github.com/posquit0/Awesome-CV
%
% Author:
% Claud D. Park <posquit0.bj@gmail.com>
% http://www.posquit0.com
%
% Template license:
% CC BY-SA 4.0 (https://creativecommons.org/licenses/by-sa/4.0/)
%

%%%%%%%%%%%%%%%%%%%%%%%%%%%%%%%%%%%%%%
%     Configuration
%%%%%%%%%%%%%%%%%%%%%%%%%%%%%%%%%%%%%%
%%% Themes: Awesome-CV
\documentclass[]{awesome-cv}
\usepackage{textcomp}
%%% Override a directory location for fonts(default: 'fonts/')
\fontdir[fonts/]

%%% Configure a directory location for sections
\newcommand*{\sectiondir}{resume/}

%%% Override color
% Awesome Colors: awesome-emerald, awesome-skyblue, awesome-red, awesome-pink, awesome-orange
%                 awesome-nephritis, awesome-concrete, awesome-darknight
%% Color for highlight
% Define your custom color if you don't like awesome colors
\colorlet{awesome}{awesome-red}
%\definecolor{awesome}{HTML}{CA63A8}
%% Colors for text
%\definecolor{darktext}{HTML}{414141}
%\definecolor{text}{HTML}{414141}
%\definecolor{graytext}{HTML}{414141}
%\definecolor{lighttext}{HTML}{414141}

%%% Override a separator for social information in header(default: ' | ')
%\headersocialsep[\quad\textbar\quad]
\begin{document}

%%%%%%%%%%%%%%%%%%%%%%%%%%%%%%%%%%%%%%
%     Profile
%%%%%%%%%%%%%%%%%%%%%%%%%%%%%%%%%%%%%%
\begin{center}
    \headerfirstnamestyle{Nishant} \headerlastnamestyle{Singh} \\
    \vspace{2mm}
    {\faEnvelope\ talktonishantsingh.ns@gmail.com} | {\faMobile\ +91 6352403156} | {\faMapMarker\ Bhopal, MP} | {\faLinkedinSquare\ \href{https://www.linkedin.com/in/tesla59/}{tesla59}} | {\faGithubSquare\ \href{https://www.github.com/tesla59/}{tesla59}} \
\end{center}
\vspace{-3mm}
%%%%%%%%%%%%%%%%%%%%%%%%%%%%%%%%%%%%%%
%     Education
%%%%%%%%%%%%%%%%%%%%%%%%%%%%%%%%%%%%%%
\cvsection{Education}
\begin{cventries}
    \cventry
    {B.Tech in Information Technology | GPA: 8.76 }
    {Indian Institute of Information Technology Bhopal}
    {Bhopal, MP}
    {Dec, 2021 – Present}
    % {GPA: 8.65}
    {}
\end{cventries}
\vspace{-9mm}

%%%%%%%%%%%%%%%%%%%%%%%%%%%%%%%%%%%%%%
%     Experience
%%%%%%%%%%%%%%%%%%%%%%%%%%%%%%%%%%%%%%
\cvsection{Work Experience}
\vspace{-3mm}
\begin{cventries}
    \vspace{-1mm}
    % -2 GSoC
    \cventry
    {Mentor}
    {Google Summer of Code}
    {Remote}
    {Mar, 2025 – Present}
    {\begin{cvitems}
        \item Mentored contributors to design \& implement a scalable observability layer for KubeArmor, integrating 6+ Prometheus metrics to expose real-time insights into policy enforcement, alerting, \& overall security activity across Kubernetes \& non-Kubernetes workloads
        \item Led and supported contributors through technical decision-making around Go development, Prometheus integration, and Kubernetes design patterns, fostering a collaborative environment while aligning implementations with CNCF’s engineering standards
        \end{cvitems}}
    % -1 Samsung
    \cventry
    {Software Developer Intern}
    {Samsung Research Institute - Delhi}
    {Noida, UP}
    {Jan, 2025 – Present}
    {\begin{cvitems}
        \item Enforced API request validations by defining field constraints (e.g., min/max values) in Swagger specs and implementing runtime checks using the validator package in Go across 10–12 backend APIs, contributing to robust API validation and improved reliability
        \item Revamped UMD movie matching by manually mapping unmatched metadata entries between Samsung’s UMD and IMDb, and developing a custom CLI tool that reduced the task time from 4–5 hours to just 30 minutes daily
        \end{cvitems}}
    % 0. LFX
    \cventry
    {LFX Mentee | Open Source Contributor}
    {Cloud Native Computing Foundation (CNCF)}
    {Salt Lake, USA}
    {Sept, 2024 – Nov, 2024}
    {\begin{cvitems}
        \item Improved the security and management functionality of KubeArmor for virtual machines (VMs) by expanding its feature set by 15\% and enhancing the capabilities for unorchestrated containers and non-kubernetes environments
        \item Developed and optimized 3 real-time configuration interfaces (via karmor.yaml, gRPC, and CLI), enabling real-time updates to security settings without the need for service restarts. Increased configuration efficiency and flexibility, reducing manual intervention
        \item Streamlined the installation and policy management process by refining the karmor CLI for VMs, reducing installation time by 25\%, while also extending CLI capabilities to enable faster policy management, resulting in a 20\% reduction in operational overhead
        \end{cvitems}}
    % \vspace{-3mm}
    % 1. Gocomet
    \cventry
    {DevOps Engineer Intern}
    {GoComet Pvt Ltd}
    {Bangalore}
    {Oct, 2023 – Oct, 2024}
    {\begin{cvitems}
            \item {Established a robust monitoring infrastructure by configuring alerts for PostgreSQL 13, specifically targeting the long idling transactions, locks and max concurrent connections, providing immediate visibility into potential issues and enabling swift resolution}
            \item {Spearheaded the daily deployment process of production code through a enhanced rolling deployment strategy on Kubernetes, orchestrating seamless updates across 5+ systems and leveraging automated tools to minimize downtime \& enhance system reliability}
            \item {Enhanced operational insights by setting up Fluentd logging agent to push detailed, enriched application logs to New Relic, facilitating comprehensive analysis and streamlined troubleshooting the intricacies and heightened understanding of application behavior}
        \end{cvitems}}
    % 2. Mobilics
    % \vspace{-3mm}
    % \cventry
    % {DevOps Engineer Intern}
    % {Mobilicis India Pvt Ltd}
    % {Remote}
    % {Jun, 2023 – Oct, 2023}
    % {\begin{cvitems}
    %         \item {Executed a successful comprehensive migration involving the seamless transition of 300+ repositories from Bit Bucket to GitHub}
    %         \item {Implemented comprehensive testing methodologies and conducted over 50 robust tests to guarantee the establishment of secure and enabled hierarchical access controls for the repositories and the CI/CD pipeline, thereby enhancing overall development efficiency}
    %         \item {Pioneered successful implementation of 8+ robust Jenkins-based CI/CD pipeline, automating build, test, and deployment processes}
    %     \end{cvitems}}

    % 3. Renal Project (Goodbye Soldier! You're the reason I am here today. You will always be remembered for being the stepping stone!)
    % \vspace{-3mm}
    % \cventry
    % {DevOps Engineer Intern}
    % {The Renal Project}
    % {Remote}
    % {Jul, 2022 – Apr, 2023}
    % {\begin{cvitems}
    %         \item {Led the design, implementation, and maintenance of automated CI/CD pipelines for code deployment using GitHub Actions, resulting in 30\% improvement in development and testing efficiency by strategically eliminating 70\% of manual deployment processes}
    %         \item {Administered successful migration of 4+ servers from Oracle Cloud Infra to AWS Cloud, employed Terraform to implement declarative methodologies and facilitated the adoption of AWS Elastic Kubernetes Service (EKS), replacing traditional Docker deployment}
    %     \end{cvitems}}
\end{cventries}
\vspace{-6mm}

%%%%%%%%%%%%%%%%%%%%%%%%%%%%%%%%%%%%%%
%     Projects
%%%%%%%%%%%%%%%%%%%%%%%%%%%%%%%%%%%%%%
\cvsection{Projects}
\vspace{-4mm}
\begin{cventries}
    % 0. GoLoadBalancer
    \cventry
    {Go, Docker, Load Balancer, Concurrency}
    {GoLoadBalancer}
    {\href{https://github.com/tesla59/GoLoadBalancer}{tesla59/GoLoadBalancer}}
    {}
    {\begin{cvitems}
            \item Developed GoLoadBalancer, an automatic load balancer with round-robin functionality, inspired by Kubernetes operator architecture
            \item Leveraged Goroutine and Docker to spawn 3+ load balancer to balance the incoming load and 5+ workers to handle incoming request
            \item Incorporated SQLite 3 Database to store and provide real-time statistics such as of all workers such as average failure rate and delay
        \end{cvitems}}

    \vspace{-5mm}
    % 1. WhisperWeb
    \cventry
    {Go, Fiber, REST APIs, SQLite, Docker}
    {WhisperWeb}
    {\href{https://github.com/tesla59/whisperweb-backend}{tesla59/whisperweb-backend}}
    {}
    {\begin{cvitems}
            \item Coded a HTTP Server using Fiber web framework in Go for back-end using SQLite 3 as the DBMS to store and retrieve data
            \item Designed the front-end part using Next JS 13 capitalising Server-Side Rendering (SSR) pattern to enhance performance and experience
            \item Containerized the website using Docker ensuring portability and reliability across different environments and easy deployment
        \end{cvitems}}

    \vspace{-5mm}
    % 2. Shell.js
    % \cventry
    % {Node JS, Linux, Bash, REPL}
    % {Shell JS}
    % {\href{https://github.com/tesla59/shell.js}{tesla59/shell.js}}
    % {}
    % {\begin{cvitems}
    %         \item Innovated and developed a custom Bash-like Shell using Node.js version 16, using built-in asynchronous libraries in JavaScript
    %         \item Redesigned asynchronous libraries to optimise performance, enabling concurrent execution of commands like CD, LS, PWD etc
    %     \end{cvitems}}

    % \vspace{-5mm}
    % 3. Hydra Kernel
    % \cventry
    % {C, Linux, Linux Kernel}
    % {Hydra Kernel}
    % {\href{https://github.com/tesla59/hydra_kernel}{tesla59/hydra\textunderscore{}kernel}}
    % {}
    % {\begin{cvitems}
    %         \item Acted as an Open-Source Developer and modified the Android Kernel based on v4.14 for Redmi Note 7 Pro for enhanced efficiency
    %         \item Increased performance by 20\% in benchmark and reduced battery consumption by 10\% in daily usage in comparison to stock kernel
    %     \end{cvitems}}


    \vspace{-3mm}
\end{cventries}

%%%%%%%%%%%%%%%%%%%%%%%%%%%%%%%%%%%%%%
%     Skills
%%%%%%%%%%%%%%%%%%%%%%%%%%%%%%%%%%%%%%
\cvsection{Skills}
\vspace{-5mm}
\begin{cventries}
    \cventry
    {}
    {
        \def\arraystretch{1.15}
        {
            \begin{tabular}{ l l }
                Languages:  & {\skill{ Go, Rust, TypeScript, JavaScript, Ruby, Dart, Python, Kotlin, C, C++}}        \\
                Tools:      & {\skill{ Docker, Kubernetes, Terraform, Ansible, Git, ArgoCD, SQL, Jenkins}}           \\
                Frameworks: & {\skill{ Flutter, React JS, Gin, GoFiber, Gorm, Express JS, Next JS}}                  \\
                Platforms:  & {\skill{ Linux, AWS Cloud, Azure, Google Cloud Platform, Oracle Cloud Infrastructure}} \\
            \end{tabular}}}
    {}
    {}
    {}
\end{cventries}
\vspace{-12mm}

%%%%%%%%%%%%%%%%%%%%%%%%%%%%%%%%%%%%%%
%     Certificates
%%%%%%%%%%%%%%%%%%%%%%%%%%%%%%%%%%%%%%
\cvsection{Certificates}
\begin{cvhonors}
    \cvhonor
    {Google Cloud Certified Professional Cloud Architect}
    {Mastered certification for the Google Cloud Professional Cloud Architect exam, showcasing expertise in architecting secure and complex solutions with high scalability and availability on the Google Cloud Platform}
    {Google Cloud}
    {Jan, 2024}
    \cvhonor
    {Google Cloud Certified Associate Cloud Engineer}
    {Achieved certification as a Google Cloud Associate Cloud Engineer, showcasing proficiency in designing, implementing and managing cloud solutions on the Google Cloud Platform and mastery in skills within Google Cloud}
    {Google Cloud}
    {Nov, 2023}
\end{cvhonors}
\
\end{document}
